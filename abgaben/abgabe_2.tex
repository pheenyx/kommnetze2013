\section{1. Übertragungsraten}
\paragraph{Aufgabe:}

	 Berechnen Sie die maximale Übertragungsrate für ein typisches Telefonsystem nach Shannon. Der Rauschabstand beträgt 30 dB, die Bandbreite beträgt 3 kHz.

\paragraph{Antwort:}

	$10 \cdot \log_{10}(\frac{S}{N})=30\, dB$
	 $\Longrightarrow \frac{S}{N} = 1000$
	\\Maximale Anzahl von $\frac{\text{Bits}}{\text{s}} =3000\text{Hz} \cdot \log_{2}(1+\underbrace{\frac{S}{N}}_\text{1000})  \approx 29902 \frac{\text{Bits}}{\textbf{s}}$


\section{2. Leitungscodes: Scrambling}
\paragraph{Aufgabe:}
		Für das Senden der Bitfolge „$1000.0100$“ verwendet der Sender NRZ als Leitungscode. Wegen der Ungleichverteilung der Bits „$1$“ und „$0$“, ergibt sich eine nicht gleichspannungsfreie Signalfolge.
		Wandeln Sie daher die gegebene Bitfolge mittels „Scambling“ in eine gleichspannungsfreie Signalfolge um (siehe dazu Folie 24).\\
		Die Pseudozufallsfolge erzeugen Sie mit $x_{i+1} := (a \cdot x_i+ b) \mod 2$, wobei $x_0$ der Startwert ist und $a$ und $b$ zwei selbst definierte Konstanten sind. \\
		Achten Sie darauf, dass die zu übertragene Signalfolge wirklich gleichspannungsfrei ist.

\paragraph{Antwort:}

Originalfolge:
$f_o =1000.0100$

Pseudozufallsfolge:
$f_p = 1010.1010$

Resultierende Folge:
$f_r = f_o \oplus f_p = 0010.1110$

Folge:
\begin{eqnarray*}        
\text{Startwert }0:&&\\
a &=& 1\\
b &=& 1\\
x0 &=& 1\\
x1 &=& 0\\
x2 &=& 1\\
x3 &=& 0\\
x4 &=& 1\\
x5 &=& 0\\
x6 &=& 1\\
x7 &=& 0\\
\end{eqnarray*}