\section{Aufgabe 1}

\begin{tabular}{|l|l|l|l|l|}\hline
Adresse hex. & Feld & Wert hex. & Wert dez. & Bedeutung \\\hline\hline
00-05 & DestMAC & 00 17 95 9A 55 67 & nicht sinnvoll & OUI Cisco \\\hline
06-0B & SourceMAC & 00 60 97 D8 EE 48 & nicht sinnvoll & OUI 3COM \\\hline
0C-0D & Length/Type & 08 00 & 2048 & InternetIP \\\hline\hline
0E.1 & Version & 4 & 4 & IPv4 \\\hline
0E.2 & IPHeaderLength & 5 & 5 & 5 byte Headerlänge \\\hline
0F & TypeOfService & 00 & 0 & no high priority\\\hline
10-11 & TotalLength & 00 30 & 48 & Bytes \\\hline
12-13 & Identification & 3A B2 & 15026 & \\\hline
14.1 & Fragments & 4 & 4 & 4 Fragmente \\\hline
14.2-15 & FragmentOffset & 0 00 & 0 & kein Offset = erstes Fragment\\\hline
16 & TimeToLive & 80 & 128 & max. 128 Hops \\\hline
17 & Protocol & 06 & 6 & TCP \\\hline
18-19 & HeaderChecksum & 4D E7 & 19943 & \\\hline
1A-1D & SourceIPAddress & 86 5D XX XX & 134.93.x.x & Rechner an der Uni MZ \\\hline
1e-21 & DestIPAddress & 86 5D XX XX & 134.93.x.x & Rechner an der Uni MZ \\\hline\hline
22-23 & IPData/SrcPort & 04 16 & 1046 & WebFilter Remote Monitor (IANA) \\\hline
24-25 & DestPort & 00 16 & 22 & Standard-SSH-Port \\\hline
26-29 & SequenceNumber & 37 8E 51 7B & 932073851 & \\\hline
2A-2D & AckNumber & 00 00 00 00 & 0 & Erstes Packet im Handshake \\\hline
2E & TCPHeaderLength & 70 & 112 & 112 * 4 Bytes\\\hline
2F & Flags & 02 & nicht sinnvoll & SYN-Flag \\\hline
30-31 & WindowSize & 40 00 & 16384 & Bytes \\\hline
32-33 & Checksum & 43 BB & 17339 & \\\hline
34-35 & UrgentPointer & 00 00 & 0 & not urgent \\\hline
36 & Options/OptionKind & 02 & 2 & MaxSegmentSize \\\hline
37 & OptionLength & 04 & 4 & Bytes \\\hline
38-39 & MaxSegmentSize & 05 B4 & 1460 & Octets \\\hline
3A-3B & OptionKind & 01 01 & 1 & NoOperation \\\hline
3C & OptionKind & 04 & 4 & SACKPermitted \\\hline
3D & OptionLength & 02 & 2 & \\\hline
\end{tabular}

\section{Aufgabe 2}
Es handelt sich um ein fragmentiertes Packet, da im Headerfeld $14$ die Anzahl der Fragmente mit $4$ angegeben wird.

\section{Aufgabe 3}
Das Segment dient dazu einen Handshake mit einem SSH-Server an der Uni Mainz aufzubauen. Es handelt sich aufgrund des SYN-Flags und der nicht vorhandenen ACK-Sequence um das erste Packet im Handshake und in dieser Verbindung. Die Verbindung wird auch von einem Rechner der Uni Mainz aufgebaut, möglicherweise sogar derselbe.
