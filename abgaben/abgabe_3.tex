\section{1. Leitungscodes}
\paragraph{Aufgabe 1.1:}
	Kodieren Sie den angegebene Bitsequenz mittels normalem Manchester-Code und geben Sie die resultierenden Signalfolge an.

\includegraphics[scale=0.2]{blatt3_1_1}
\paragraph{Aufgabe 1.2:}
	Kodieren Sie den angegebene Bitsequenz mittels differentiellem Manchester-Code und geben Sie die resultierenden Signalfolge an. Gibt es eine eindeutige Lösung ?
\includegraphics[scale=0.2]{blatt3_1_2}

\paragraph{Aufgabe 1.3:}
	Diskutieren und vergleichen (Tip: Rauschen) Sie kurz die Eigenschaften der beiden Leitungscodes.
\paragraph{Antwort:}
\section{2. Codemultiplex}
Gegeben seien die Walsh-Hadamard-Sequenzen der Länge 8 aus dem Skript.\\
Zu senden sind an den Empfänger 7 die Bitfolge $110$ und an den Empfänger 3 die Bitfolge $101$.
\paragraph{Aufgabe 1.1:}
	Berechnen Sie die zu sendenden Sendefolgen.
\paragraph{Antwort:} Die zu sendende Bitfolge ergibt sich aus der Addition beider Bitfolgen
	\begin{enumerate}

	\item Die Bitfolge für Empfänger $3$
	
		\begin{tabular}{|c||c|c|c|c|c|c|c|c|}
		\hline 1: & -1 & -1 & +1 & +1 & -1 & -1 & +1 & +1 \\ 
		\hline 0: & +1 & +1 & -1 & -1 & +1 & +1 & -1 & -1 \\ 
		\hline 1: & -1 & -1 & +1 & +1 & -1 & -1 & +1 & +1 \\ 
		\hline 
		\end{tabular} 
	
	\item Die Bitfolge für Empfänger $7$
		
		\begin{tabular}{|c||c|c|c|c|c|c|c|c|}
		\hline 1: & -1 & -1 & +1 & +1 & +1 & +1 & -1 & -1 \\ 
		\hline 1: & -1 & -1 & +1 & +1 & +1 & +1 & -1 & -1 \\ 
		\hline 0: & +1 & +1 & -1 & -1 & -1 & -1 & +1 & +1 \\ 
		\hline 
		\end{tabular} 
		
	\item Die Addierte Bitfolge ist demnach
	
		\begin{tabular}{|c|c|c|c|c|c|c|c|}
		\hline -1-1 & -1-1 & +1+1 & +1+1 & -1+1 & -1+1 & +1-1 & +1-1 \\
		$\Downarrow$ & $\Downarrow$ & $\Downarrow$ & $\Downarrow$ & $\Downarrow$ & $\Downarrow$ & $\Downarrow$ & $\Downarrow$ \\
		-2 & -2 & +2 & +2 & 0 & 0 & 0 & 0 \\ 
		\hline +1-1 & +1-1 & -1+1 & -1+1 & +1+1 & +1+1 & -1-1 & -1-1 \\
		$\Downarrow$ & $\Downarrow$ & $\Downarrow$ & $\Downarrow$ & $\Downarrow$ & $\Downarrow$ & $\Downarrow$ & $\Downarrow$ \\
		0 & 0 & 0 & 0 & +2 & +2 & -2 & -2 \\ 
		\hline -1+1 & -1+1 & +1-1 & +1-1 & -1-1 & -1-1 & +1+1 & +1+1 \\ 
		$\Downarrow$ & $\Downarrow$ & $\Downarrow$ & $\Downarrow$ & $\Downarrow$ & $\Downarrow$ & $\Downarrow$ & $\Downarrow$ \\
		0 & 0 & 0 & 0 & -2 & -2 & 2 & 2 \\
		\hline 
		\end{tabular} 
	
	\end{enumerate}
\paragraph{Aufgabe 1.2:}
	Berechnen Sie aus den empfangenen Sendefolgen die ursprüngliche Bitfolge.
\paragraph{Antwort:} Um die gesendeten Sendefolgen zu entschlüsseln müssen wir jetzt einfach das Skalarprodukt aus empfangener Folge und eigenem Schlüssel bilden.
	\begin{enumerate}
		\item Empfänger $3$
		
			\begin{tabular}{|c|c|c|c|c|c|c|c|c|c|}
			\hline empfangene Folge & -2 & -2 & 2 & 2 & 0 & 0 & 0 & 0 &  \\ 
			$\times$ eigener Schlüssel & 1 & 1 & -1 & -1 & 1 & 1 & -1 & -1 &  \\ 
			= & -2 & -2 & -2 & -2 & 0 & 0 & 0 & 0 & $\sum = -8 \Rightarrow \text{gesendetes Bit: 1}$ \\ 
			\hline empfangene Folge & 0 & 0 & 0 & 0 & 2 & 2 & -2 & -2 &  \\ 
			$\times$ eigener Schlüssel & 1 & 1 & -1 & -1 & 1 & 1 & -1 & -1 &  \\ 
			= & 0 & 0 & 0 & 0 & 2 & 2 & 2 & 2 & $\sum = 8 \Rightarrow \text{gesendetes Bit: 0}$ \\ 
			\hline empfangene Folge & 0 & 0 & 0 & 0 & -2 & -2 & 2 & 2 &  \\ 
			$\times$ eigener Schlüssel & 1 & 1 & -1 & -1 & 1 & 1 & -1 & -1 &  \\ 
			= & 0 & 0 & 0 & 0 & -2 & -2 & -2 & -2 & $\sum = -8 \Rightarrow \text{gesendetes Bit: 1}$ \\ 
			\hline 
			\end{tabular} 
		
		\item Empfänger $7$	
		
			\begin{tabular}{|c|c|c|c|c|c|c|c|c|c|}
			\hline empfangene Folge & -2 & -2 & 2 & 2 & 0 & 0 & 0 & 0 &  \\ 
			$\times$ eigener Schlüssel & 1 & 1 & -1 & -1 & -1 & -1 & 1 & 1 &  \\ 
			= & -2 & -2 & -2 & -2 & 0 & 0 & 0 & 0 & $\sum = -8 \Rightarrow \text{gesendetes Bit: 1}$ \\ 
			\hline empfangene Folge & 0 & 0 & 0 & 0 & 2 & 2 & -2 & -2 &  \\ 
			$\times$ eigener Schlüssel & 1 & 1 & -1 & -1 & -1 & -1 & 1 & 1 &  \\ 
			= & 0 & 0 & 0 & 0 & -2 & -2 & -2 & -2 & $\sum = -8 \Rightarrow \text{gesendetes Bit: 1}$ \\ 
			\hline empfangene Folge & 0 & 0 & 0 & 0 & -2 & -2 & 2 & 2 &  \\
			$\times$ eigener Schlüssel & 1 & 1 & -1 & -1 & -1 & -1 & 1 & 1 &  \\ 
			= & 0 & 0 & 0 & 0 & 2 & 2 & 2 & 2 & $\sum = 8 \Rightarrow \text{gesendetes Bit: 0}$ \\ 
			\hline 
			\end{tabular} 
		
	\end{enumerate}