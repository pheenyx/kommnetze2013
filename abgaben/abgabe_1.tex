\section{Aufgabe 1:}
Diskutieren Sie kurz folgende Eigenschaften für jede Topologie: Grad,
Durchmesser, Bisektionsweite, Congestion.
\subsection{Bus Topologie}
In einem Bus sind die Stationen nur passiv abnehmer der Leitung.
\begin{itemize}
  \item Grad: $1$
  \newline Es gibt nur eine Kante auf denen die Knoten passiv draufsitzen. Daher
  ist der Grad $1$. Jeder Knoten hat nur Eine und zwar genau eine ausgehende
  Kante.
  \item Durchmesser: $1$
   \newline Der längste kürzeste Weg zwischen zwei Knoten ist $1$, da die
   Nachricht einfach abgelesen werden kann und sie nicht durch andere Knoten
   hindurch muss da die Knoten nur passive Funktion haben.
  \item Bisektionsweite: $1$
  \newline  Um das Netz mit $N$ Knoten in zwei Netze mit jeweils $\frac{N}{2}$
  Knoten zu teilen, muss man $1$ Kanten durchschneiden.
  \item Congestion: $1$
  \newline Da der Bus nur aus einer Kante besteht ist die Congestion dem
  entsprechend $1$.
\end{itemize}

\subsection{Stern Topologie}
Ein Sternnetz besitzt einen zentralen Knoten mit $N-1$ Nachbarknoten.

\begin{itemize}
  \item Grad: $1$ \newline Die Anzahle der ausgehenden Verbindungen aller
  Knoten, die nicht zentral sind, ist $1$.
  \item Durchmesser: $2$ \newline Jede Nachricht muss über den Zentralknoten und
  kommt von da direkt zum Zielknoten. Daher sind zwei Hops für jede Nachricht
  notwendig.
  \item Bisektionsweite: $\frac{N}{2}$ \newline Um das Netz in zwei Sets an
  Knoten zu trennen, müssen vom zentralen Knoten $\frac{N}{2}$ Knoten getrennt
  werden.
  \item Congestion: $1$ \newline Für jeden Nachricht gibt es maximal eine Weg
  über den zentralen Knoten.
\end{itemize}

\subsection{Baum Topologie}
Ein Baum ist ein Verbundnetz aus beliebig vielen Stern- und Busnetzwerken.
\begin{itemize}
  \item Grad: $1$
  \newline Da sowohl Stern- als auch Busnetze einen Grad von $1$ haben, hat ein
  daraus aufgebauter Baum auch einen Grad von $1$.
  \item Durchmesser: $N-1$ ($N =$ Anzahl der Knoten im Graphen)
   \newline Ein entarteter Baum, der aus Sternnetzen aufgebaut ist, die je einen
   eingehenden (vom Vaterknoten) und einen ausgehenden (zum Kindknoten) Knoten
   haben, besteht aus einem Ast. Dieser hat $N-1$ Kanten.
  \item Bisektionsweite: 1
  \newline Ein Baum lässt sich an der Wurzel in zwei Sets an Knoten trennen.
  Hierbei muss nur eine Kante getrennt werden.
  \item Congestion: 1
  \newline Da ein Baum per Definition zyklenfrei ist, kann es nur einen Weg
  zwischen zwei Knoten geben, daher ist die Congestion $1$.
\end{itemize}

\subsection{Ring Topologie(eine Richtung)}
 Jeweils zwei Teilnehmer sind über Einpunktverbindungen miteinander verbunden.
\begin{itemize}
  \item Grad: $1$
  \newline Eine Nachricht bewegt sich in einer Richtung von Knoten zu Knoten. 
  Sie kann nur in eine Richtung gesendet werden. Daraus folgt der Graph ist ein
  gerichteter Graph. Da jeder Knoten nur einen direkten Nachbarn besitzt, ist
  der Grad $1$.
  \item Durchmesser: $N-1$ ($N =$ Anzahl der Knoten im Graphen.)
   \newline Der längste kürzeste Weg zwischen zwei Knoten ist $N-1$, da die
   Nachricht im schlimmsten Fall einmal durch den ganzen Ring durchgeleitet
   werden muss.
  \item Bisektionsweite: $2$
  \newline  Um das Netz mit $N$ Knoten in zwei Netze mit jeweils $\frac{N}{2}$
  Knoten zu teilen, muss man $2$ Kanten durchschneiden.
  \item Congestion: $1$
  \newline Da sich die Nachrichten in einem Ring-Netzwerk in einer Richtung von
  Knoten zu Knoten bewegen.
\end{itemize}

\subsection{Ring Topologie}
 Jeweils zwei Teilnehmer sind über Zweipunktverbindungen miteinander verbunden.
 Es handelt sich hierbei im einen bidirektionalen Ring.
\begin{itemize}
  \item Grad: $2$
  \newline Eine Nachricht bewegt sich in einer Richtung von Knoten zu Knoten.
  Sie kann entweder nach Rechts oder Links wandern. Daraus folgt der Graph ist
  ein ungerichteter Graph. Da jeder Knoten zwei Nachbarn besitzt, ist der Grad 2.
  \item Durchmesser: $\frac{N}{2}$ ($N =$ Anzahl der Knoten im Graphen.)
   \newline Der längste kürzeste Weg zwischen zwei Knoten ist $\frac{N}{2}$, da
   die Nachricht entweder nach links oder rechts wandern kann.
  \item Bisektionsweite: $2$
  \newline  Um das Netz mit $N$ Knoten in zwei Netze mit jeweils $\frac{N}{2}$
  Knoten zu teilen, muss man $2$ Kanten durchschneiden.
  \item Congestion: $1$
  \newline Da sich die Nachrichten in einem Ring-Netzwerk in einer Richtung von
  Knoten zu Knoten bewegen.
\end{itemize}

\subsection{Vollvermaschtes Netz}
Der Graph ist vollständig. Die Knoten sind über Punkt-zu-Punkt-Verbindungen
miteinander verbunden.
\begin{itemize}
  \item Grad: $N-1$
  \newline Der Graph ist vollständig. Jeder Knoten hat $N-1$ ausgehende Kanten.
  \item Durchmesser: $1$
   \newline Die Knoten sind über Punkt-zu-Punkt-verbindungen miteinander
   verbunden. Der längste kürzeste Weg  im Graphen hat Länge $1$ Hop.
  \item Bisektionsweite: $ \left( \frac{N}{2}\right)^2$
  \newline Man teilt den Graphen in $2$ Netze mit jeweils $\frac{N}{2}$ Knoten.
  Da der Graph vollständig ist, hat jeder Knoten eines Netzes  $\frac{N}{2}$
  ausgehende Kanten. d.h ein Netz hat $\frac{N}{2} \, \cdot \, \frac{N}{2}$
  ausgehende Kanten.
  \item Congestion: $N-1$
  \newline Es gibt $N-1$ mögliche Wege um eine Nachricht zwischen zwei Knoten zu
  schicken.
\end{itemize}


\section{Aufgabe 2:}
Berechnen Sie die Übertragungsdauer bei gegebener Übertragungsrate
\begin{itemize}
\item 100 MB bei 100 MBit/s \\
\begin{equation*}
 \frac{100 \cdot 1024^2 \, \textrm{bytes}}{\frac{100}{8} \cdot 1000^2 \, \frac{\textrm{Bit}}{\textrm{s}}} = 8,388608 \, \textrm{s} 
\end{equation*}

\item 10 GB bei 100 MBit/s \\
\begin{equation*}
 \frac{10 \cdot 1024^3 \, \textrm{bytes}}{\frac{100}{8} \cdot 1000^2 \, \frac{\textrm{Bit}}{\textrm{s}}} = 858,9934592 \, \textrm{s} = 14,31655765\overline{3} \, \textrm{min}
\end{equation*}

\item 10 GB bei 1 GBit/s \\
\begin{equation*}
 \frac{10 \cdot 1024^3 \, \textrm{bytes}}{\frac{1}{8} \cdot 1000^3 \, \frac{\textrm{Bit}}{\textrm{s}}} = 85,89934592 \, \textrm{s} = 1,431655765\overline{3} \, \textrm{min}
\end{equation*}

\item 10 GB bei 10 GBit/s \\
\begin{equation*}
 \frac{10 \cdot 1024^3 \, \textrm{bytes}}{\frac{10}{8} \cdot 1000^3 \, \frac{\textrm{Bit}}{\textrm{s}}} = 8,589934592 \, \textrm{s}
\end{equation*}

\item 100 GB bei 10 GBit/s \\
\begin{equation*}
 \frac{100 \cdot 1024^3 \, \textrm{bytes}}{\frac{10}{8} \cdot 1000^3 \, \frac{\textrm{Bit}}{\textrm{s}}} = 85,89934592 \, \textrm{s} = 1,431655765\overline{3} \, \textrm{min}
\end{equation*}

\item 200 MB bei 10 GBit/s \\
\begin{equation*}
 \frac{200 \cdot 1024^2 \, \textrm{bytes}}{\frac{10}{8} \cdot 1000^3 \, \frac{\textrm{Bit}}{\textrm{s}}} = 0,16777216 \, \textrm{s}
\end{equation*}


\end{itemize}
