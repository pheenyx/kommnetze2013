\section{Aufgabe 1}

Beim CSMA/CD horcht eine sendewillige Station das Medium ab, bevor sie anfängt
zu senden. Ist das Medium noch belegt, wird die Station warten bis es frei wird.

Sollte das MEdium frei sein, entscheidet die X-persistent Strategie, ob die
Station sofort anfängt zu senden oder nicht.

Bei einer 1-persistent Strategie fängt die Station mit einer Wahrscheinlichkeit
von $1$ an zu senden. Bei einer p-persistent Strategie nur mit einer
Wahrscheinlichkeit von $p$. Ansonsten wartet sie einen Slot. Eine non-persitent
Strategie wartet eine zufällige Zeit ab, um zu senden.

Wenn beim Senden eine Kollision auftritt können das alle an das Medium
angeschlossene Stationen erkennen, da der Spannungspegel des Signals durch
positive Überlagerung über einen Grenzwert tritt. Die Station, die das erkennt
sendet ihrerseits ein JAM-Signal. Dieses Signal verhindert, dass eine Station
das Packet akzeptiert, da es sicher die CRC des Packetes ungültig macht.

Damit die ausgehend sendende Station erkennt, dass das Packet nicht richtig
gesendet wurde, muss das JAM-Signal den Sender erreichen, bevor der Frame zuende
gesendet wurde.


\section{Aufgabe 2}

Ein Token-Ring ist ein System aus zu einem Ring zusammengeschlossen Rechner. Das
Netz wird aus Verbindungen zwischen jeweils zwei Rechnern aufgebaut. Damit
existiert kein Broadcastmedium im eigentlichen Sinne.

Damit eine Station senden kann, benötigt sie das sog. Token. Ein Token ist eine
Nachricht, die das sendewillige System vom Ring "nehmen" kann, wenn die
Prioritäten richtig gesetzt sind. Vom Ring nehmen bedeutet hierbei, dass die
Station das Token nicht weiter sendet. 

Hat eine Station das Token, kann sie Nachrichten senden. Hierbei wird eine
Nachricht in eine Richtung auf den Ring gesetzt und von jeder folgenden Station
weiter gesendet. Die Zielstation nimmt die Nachricht an und verarbeitet sie,
sendet sie allerdings auch weiter. Erst die sendende Station nimmt die Nachricht
wieder vom Ring. Damit ist gewährleistet, dass die Nachricht von alle möglichen
Empfänger gelesen werden kann. Des weiteren kann damit überprüft werden, ob der
Ring noch korrekt funktioniert.

Nachdem eine oder mehrere Nachrichten gesendet wurden, setzt die sendende
Station ihrerseits ein neues Token auf den Ring, dass von der nächsten Station
mit ausreichender Priortät genommen wird.

Eine Station des Rings übernimmt die Aufgabe als aktiver Monitor. Dieser nimmt
endlos zirkulierende Rahmen oder Token mit zu hoher Priorität vom Ring und
erstetzt verloren gegangene oder geglaubte Token. Desweiteren schützt er davor,
dass mehrere Monitore gleichzeitig versuchen diese Aufgabe zu übernehmen.
