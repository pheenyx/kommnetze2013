\section{Aufgabe 1}

\subsection{1.}
Maximale Wartezeit sind $2^{x}-1$ Slots, hier also $2^8-1$ Slots, ergo $255$.

\subsection{2.}
S.\,o.\,. Hierbei muss die Station maximal $511$ Slots warten.

\subsection{4.}
Es handelt sich um 100 MBit Ethernet.
\[ 
    1 \text{ Slot } = 5,12 \mu s \text{ (RTT als minimale Slotgröße)}
\]
\[ 
    30,72 \mu s = 6 \text{ Slots}
\]
\[  
    log_2(6) = 2,\ldots \Rightarrow 3 \text{ Versuche}
\]


\section{Aufgabe 2}

\section{Aufgabe 3}

\paragraph{Nachteile Datagramme:}
\begin{enumerate}
\item Hoher Mehraufwand für die Adressierung während der Datenübertragung
\item evtl. niedrige Qualität des ankommenden Paketstroms:  Neusortierung oder Fehlerüberwachung in den höheren Schichten nötig
\item ??
\end{enumerate}

\paragraph{Nachteile Virtuelle Verbindung:}
\begin{enumerate}
\item Schwieriger zu implementieren
\item Verbindungsaufbau und -abbau, deshalb hoher Overhead für kurzlebige Verbindungen
\item Unflexibel und nicht so zuverlässig wie Datagramme.
\end{enumerate}